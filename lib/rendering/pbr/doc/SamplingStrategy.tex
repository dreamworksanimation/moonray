\title{\textbf{MCRT Sampling Strategies and Implementation Notes}}
\author{
        \textbf{Eric Tabellion}\\
        \textit{DreamWorks Animation}
}
\date{\today}

\documentclass{article}
\usepackage{amssymb,amsmath}
%\usepackage{fullpage}
\usepackage[top=1in, bottom=1in, left=1in, right=1in]{geometry}


%------------------------------------------------------------------------------

\begin{document}

% Try to set a better font
%\renewcommand{\familydefault}{\sfdefault}
% Pick a larger font size to make it more readable
\large

\maketitle


%----------------------------------------

\section{Sampling Strategies}\label{Strategies}

Our main problem is to solve the rendering equation by Monte Carlo
integration:

\begin{equation*}
L_r(x,\omega_o)  =  \int_{\Omega}  f_r(x,\omega_o,\omega_i)  L_i(x,\omega_i) \
                   |\cos(\theta_i)|  \  d\omega
\end{equation*}

We want to use different sampling strategies at each point $x$
depending on the type of light and Bsdf lobe combination that we
encounter. In the general case, we want to first partition lights
from our light-set into one or many disjoint subsets. Then, for each
light subset, we want to partition the lobes from the Bsdf into one
or many disjoint lobe subsets (possibly differently for each light
subset). For each subset combination we then decide to use one of
the following sampling methods: Bsdf Importance Sampling (BIS),
Multiple Importance Sampling (MIS) or Light Importance Sampling
(LIS).

Table \ref{table:Strategies} lists all the combinations (light category
x bsdf lobe category) with the corresponding sampling method we want
to use \cite{Colbert:10} \cite{Hery:13} \cite{Pharr:10}
\cite{Shirley:96}. Each sampling method with its numeric suffix defines
a specific sampling strategy:

\begin{table}[ht]
\centering
\begin{tabular}{c c c c c c}
\hline\hline
Light category  &  SSS   &  Wide  &  Medium  &  Narrow  &  Delta  \\ [0.5ex]
\hline
Indirect        &  BISS  &  BISI  &  BISI    &  BISI    &  BISI   \\
Wide            &  MISS  &  MIS2  &  MIS2    &  BIS2    &  BIS1   \\
Narrow          &  LISS  &  LIS2  &  MIS1    &  BIS2    &  BIS1   \\
Delta           &  LISS  &  LIS1  &  LIS1    &  LIS1    &  none   \\ [0.5ex]
\hline
\end{tabular}
\caption{Sampling strategies for each combination of light category x bsdf lobe category}
\label{table:Strategies}
\end{table}

Based on this table, each sampling strategy defines the
subset of Bsdf lobes and lights that are used to sample from and
that contribute to its corresponding estimator. For example, the
strategy MIS2 will draw samples from the wide and medium lobes category
as well as from wide lights and only these lobes and lights will contribute
to its estimator (as defined later in section \ref{Estimators}).

The definition of wide, medium and narrow lobes may seem subjective.
Each Bsdf lobe will define which category it belongs to, based
on a threshold on its roughness parameter and on experimentation on
which sampling method works best. For example, diffuse and very blurry
glossy lobes will belong to the "wide" category. Almost mirror-like
lobes will belong to the "narrow" category, etc.

Similarly, lights will belong to either the wide or narrow category
based on their subtended solid angle from the point $x$ being shaded.
It is worth noting that both Bsdf lobes and lights will dynamically belong
to one or another category depending on the point $x$ being shaded (parameters
such as lobe roughness and light subtended solid angle are functions of $x$).


%----------------------------------------

\section{Splitting the Rendering Equation}\label{Splitting}

Let's now define formally how the above partitioning will impact our
numerical integration scheme. For the sake of clarity, we simplify
the notation from the rendering equation, by omitting the cosine
term and other details. We then split the integral into a sum of
integrals to partition the type of lighting we are integrating:

\begin{align*}
L_r &= \int f_r  L_i  \\
    &= \int f_r  L_{indirect}  +  \int f_r  L_{direct}  \\
    &= \int f_r  L_{indirect}  +  \int f_r  L_{wide}  +  \int f_r  L_{narrow}  +  \int f_r  L_{delta}  \\
    &= F_{indirect}  +  F_{wide}  +  F_{narrow}  +  F_{delta}
\end{align*}

Here $L_{indirect}$ is indirect illumination and $L_{direct}$ is
direct illumination. The latter is decomposed into
delta-distribution lights $L_{delta}$ (also called CG lights, i.e.
lights with zero area), physical lights $L_{wide}$ that belong to
the wide category and physical lights $L_{narrow}$ that belong to
the narrow category.

We then split some of these integrals further to partition the type
of Bsdf lobes that are involved in each convolution. We use a
sampling strategy subscript on the Bsdf, to indicate not only the
sampling strategy that we use for the corresponding estimator, but
also  which lobe subset is involved in each integral (referring to table
\ref{table:Strategies}):
\begin{align*}
F_{indirect}  &=  \int f_{BISI}  \  L_{indirect}  \\
F_{wide}      &=  \int f_{MIS2}  \  L_{wide}  +  \int f_{BIS2}  \  L_{wide}  +  \int f_{BIS1}  \  L_{wide}  \\
              &=  F_{wide_{MIS2}}  +  F_{wide_{BIS2}}  +  F_{wide_{BIS1}}  \\
F_{narrow}    &=  \int f_{LIS2}  \  L_{narrow}  +  \int f_{MIS1}  \  L_{narrow}  +  \int f_{BIS2}  \  L_{narrow}  +  \int f_{BIS1}  \  L_{narrow}  \\
              &=  F_{narrow_{LIS2}}  +  F_{narrow_{MIS1}}  +  F_{narrow_{BIS2}}  +  F_{narrow_{BIS1}}  \\
F_{delta}     &=  \int f_{LIS1}  \  L_{delta}
\end{align*}


%----------------------------------------

\section{Monte Carlo Estimators}\label{Estimators}

Our Monte Carlo integration approach will consist in sampling and
evaluating all of these integrals. We now write the estimators that
we will use to numerically integrate each of them.

In the equations that follow, the light contribution ($L_{sub}$) of
a Bsdf sample simply means the contribution of the closest light in
the subset, in the direction of the sample, or zero if the closest
light is not in the subset or is occluded from the point $x$ by
geometry. The light contribution of a light sample is the
contribution of that light sample or zero if it is occluded by
geometry or any other light.

We use a similar notation as in \cite{Veach:95}. $N_{L_{sub}}$ is
the number of lights in the subset, $N_{B_{sub}}$ is the number of
Bsdf lobes in the subset, and ${n_i}$ is the number of samples drawn
from each light or Bsdf lobe $i$. Finally, $vis()$ is the shadowing
function.


\vspace{.5cm}
\textbf{Estimators when sampling from the lights:}

\vspace{.2cm}
\textbf{LIS2:} Narrow lights x wide lobes.
\begin{equation}
F_{narrow_{LIS2}} =
    \sum_{i}^{N_{L_{narrow}}}  \frac{1}{n_i}  \sum_{j}^{n_i}  \sum_{k}^{N_{B_{LIS2}}}
        \frac{  L_i(X_{i,j}) }  { pdf_{L_i}(X_{i,j}) }
        f_k(X_{i,j}) . vis(X_{i,j})
        |\cos\theta_{X_{i,j}}|
\end{equation}

\textbf{LIS1:} Delta lights x non-delta lobes. The estimator is
similar to $F_{narrow_{LIS2}}$ aside for the lobe subset involved.
Since we only take a single sample per delta light and we're dealing
with delta-functions, we have $pdf_{L_i}(X_i) = 1$. The estimator
simplifies to:
\begin{equation}
F_{delta} =
    \sum_{i}^{N_{L_{delta}}}  \sum_{k}^{N_{B_{LIS1}}}
        L_i(X_i) f_k(X_i) . vis(X_i)
        |\cos\theta_{X_i}|
\end{equation}


\vspace{.5cm}
\textbf{Estimators when sampling from the Bsdf lobes:}

\vspace{.2cm}
\textbf{BIS2:} The union of wide and narrow lights x narrow lobes.
\begin{equation}
F_{wide_{BIS2}} + F_{narrow_{BIS2}} =
    \sum_{i}^{N_{B_{BIS2}}}  \frac{1}{n_i}  \sum_{j}^{n_i}
        \frac{  f_i(X_{i,j}) }  { pdf_{f_i}(X_{i,j}) }
        L_{wide \bigcup narrow}(X_{i,j}) . vis(X_{i,j})\ 
        |\cos\theta_{X_{i,j}}|
\end{equation}

\vspace{.2cm}
\textbf{BIS1:} The union of wide and narrow lights x delta lobes. The
estimator is similar to $F_{wide_{BIS2}} + F_{narrow_{BIS2}}$ aside for the
subsets involved. Since we only take a single sample from each delta
lobe and we're dealing with delta-functions, we have
$pdf_{f_{BIS1}}(X_i) = 1$. The estimator simplifies to:
\begin{equation}
F_{narrow_{BIS1}} + F_{wide_{BIS1}} =
    \sum_{i}^{N_{B_{BIS1}}}
        f_i(X_i) \ 
        L_{wide \bigcup narrow}(X_i) . vis(X_i)\ 
        |\cos\theta_{X_i}|
\end{equation}

\vspace{.2cm}
\textbf{BISI:} Indirect lighting x all lobes. Note that evaluating
$L_{indirect}$ requires tracing a ray in the direction of the sample and
evaluating $L_{indirect}$ recursively. If the ray hits a light source, then
$L_{indirect}$ evaluates to zero to avoid counting the hit light contribution
again.
\begin{equation}
F_{indirect} =
    \sum_{i}^{N_{B_{BISI}}}  \frac{1}{n_i}  \sum_{j}^{n_i}
        \frac{  f_i(X_{i,j}) }  { pdf_{f_i}(X_{i,j}) }
        L_{indirect}(X_{i,j})
        |\cos\theta_{X_{i,j}}|
\end{equation}


\vspace{.5cm}
\textbf{Estimators when sampling from both the Bsdf lobes and the lights:}

\vspace{.2cm}
\textbf{MIS1:} Narrow lights x medium lobes. We refer the reader
to \cite{Veach:95}, equation (11) for the combined estimator:
\begin{align*}
F_{narrow_{MIS1}}
    &= \sum_{i}^{N_{B_{MIS1}}}  \frac{1}{n_i}  \sum_{j}^{n_i}
        w_{f_{i}L_{narrow}}(X_{i,j})
        \frac{  f_i(X_{i,j}) }  { pdf_{f_i}(X_{i,j}) }
        L_{narrow}(X_{i,j}) . vis(X_{i,j})  |\cos\theta_{X_{i,j}}|  \\
    &+ \sum_{i}^{N_{L_{narrow}}}  \frac{1}{n_i}  \sum_{j}^{n_i}  \sum_{k}^{N_{B_{MIS1}}}
        w_{L_i f_k}(X_{i,j})
        \frac{ L_i(X_{i,j}) }  { pdf_{L_i}(X_{i,j}) }
        f_k(X_{i,j}) . vis(X_{i,j})  |\cos\theta_{X_{i,j}}|
\end{align*}
The $w_{sub1 sub2}$ can be defined generally using the balance
heuristic as in \cite{Veach:95}, equation (12) and using the power
heuristic as defined in equation (14), although the latter has been
simplified. Since we want to use the power heuristic, we can
re-write (14) without any simplification and substituting the $c_i$
by $n_i$ (this is unbiased as long as $n_i > 0$).
\begin{align*}
w_{f_{i}L_k}(x)  &=  \frac{ {(n_i * pdf_{f_i}(x))}^\beta }
    { {(n_i * pdf_{f_i}(x))}^\beta + {(n_k * pdf_{L_k}(x))}^\beta } \\
w_{L_i f_k}(x)     &=  \frac{ {(n_i * pdf_{L_i}(x))}^\beta }
    { {(n_i * pdf_{L_i}(x))}^\beta + {(n_k * pdf_{f_k}(x))}^\beta }
\end{align*}


\vspace{.2cm}
\textbf{MIS2:} Wide lights x Wide and medium lobes. This estimator is the 
same as the estimator for \textbf{MIS1}, with the following substitutions:
\begin{align*}
 & narrow \rightarrow wide \\
 & MIS1 \rightarrow MIS2
\end{align*}


%----------------------------------------

\section{Sampling Budget}\label{Budget}

Our goal is to take a single set of Bsdf samples and a single set of
light-set samples and use them to evaluate all the estimators
given in the previous section that apply in a given situation.

We compute a sample budget for each Bsdf lobe and for each light in
turn. Initially, this is driven by user-adjustable parameters at the
scene level. Parameters such as "direct bsdf samples", "indirect
bsdf samples" and "indirect subsurface samples" will be used to
derive maximum per lobe sample counts. A scene parameter "light
samples" and an optional per-light "sampling factor" attribute will
drive per-light sample budgets.

Then, the following properties may affect the sample budget further:
lobe type, albedo approximation, roughness, light type, light solid
angle, light intensity, SIMD width of the hardware, sample
stratification constraints, random number generator constraints.

\textbf{TODO:} Add more detail on how each of these parameters
affect the sample budget.

The sample budget will finally be affected by the current path
throughput. In short, the deeper the light bounce recursion, the
lower the path throughput and the lower the sample budget will be.

\textbf{TODO:} If we integrate subsurface scattering with indirect
lighting and we do have a diffuse Bsdf lobe too, we will re-use the
same samples. Otherwise we will add an additional set of samples
(with a cosine wweighted distribution) for this purpose.


%----------------------------------------

\section{Sample Indexing}\label{Indexing}

We then organize and index the samples according to the light and
Bsdf lobe partitioning, in a way that allows us to simultaneously
and efficiently compute all the needed estimators defined in section
\ref{Estimators}.

\textbf{TODO:} Add sample indexing figure.


%----------------------------------------

\section{Algorithm}\label{Algorithm}

\textbf{TODO:} Add integrator algorithm here.


%----------------------------------------

\bibliographystyle{abbrv}
\bibliography{SamplingStrategy}


%------------------------------------------------------------------------------

\end{document}


%------------------------------------------------------------------------------
%------------------------------------------------------------------------------
%------------------------------------------------------------------------------

Notes / TODOs:
--------------

// TODO: Splitting in Veach's thesis calls for having the splitting factor be
// a function of the magnitude of the estimated contribution of a sample
// (i.e. function of path throughput, total emitted power for a light, albedo
// for a bsdf ?) See [Arvo and Kirk 1990]

// TODO: determine a weight alpha for each light, based on estimated light
// contribution. This is tricky with glossy bsdf's, so set alpha = 1/lightCount
// for now

- Choose != number of samples for each sampling strategy
  - Lights:
    - Per light multiplier or min & max samples ?
    - Compare with "linear alpha" approach of [13] where alpha = estimate is bsdf * Le * geometry term ignoring visibility
    - Light culling:
      - cull luminaires / samples entirely below surface plane
      - cull further with a spatial structure [13] with Lbright / Ldim light lists in each cell (works only for diffuse)
  - Bsdf
    - Per lobe multiplier or min & max samples ?
    - Cull bsdf samples below the surface
    - See light-bvh for culling bsdf samples / testing which light they hit (for MIS) in Christophe's course notes

x Will we need a proper lobe sampling implementation for later bounces ?
  --> go with 1 sample per lobe like we do now and let Russian Roulette do its job

- Compute good LIS / BIS / MIS cutoffs numerically, based on roughness and area light size ?

- Is RIS worth it with SIMD ?
  - "reduce shadow rays, especially when lots of light samples are needed to resolve high frequency in light EDF"
  - How does it mesh with MIS ?
- [12] P. Shirley and C. Wang. Distribution ray tracing: theory and prac- tice. Proceedings of the Third Eurographics Workshop on Rendering,
Bristol, England, 33\ufffd44 (1992).
- [13] P. Shirley, C. Wang, and K. Zimmerman. Monte Carlo Techniques
for Direct Lighting Calculations. ACM Transactions on Graphics, to
appear.

